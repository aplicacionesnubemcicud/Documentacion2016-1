
%%%%%%%%%%%%%%%%%%%%%%%%%%%%%%%%%%%%%%%%%%%%%%%%%%%%%%%%%%%%%%%%%%%%%%%%%%%%
%%%%%%%%%%%%%%%%%%%%%%%%%%%%%%%%%%%%%%%%%%%%%%%%%%%%%%%%%%%%%%%%%%%%%%%%%%%%
%%%%         Texmaker 3.5.2 with MiKTeX 2.9 over Windows 7          %%%%%%%%
%%%%%%%%%%%%%%%%%%%%%%%%%%%%%%%%%%%%%%%%%%%%%%%%%%%%%%%%%%%%%%%%%%%%%%%%%%%%
%%%%%%%%%%%%%%%%%%%%%%%%%%%%%%%%%%%%%%%%%%%%%%%%%%%%%%%%%%%%%%%%%%%%%%%%%%%%


%%%%%%%%%%%%%%%%%%%%%%%%%%%%%%%%%%%%%%%%%%%%%%%%%%%%%%%%%%%%%%%%%%%%%%%%%%%%
%%%%%%%%%%%%%%%%%%%%%%%%%%%%%%%%%%%%%%%%%%%%%%%%%%%%%%%%%%%%%%%%%%%%%%%%%%%%
%%%%            Please Compile the Program Twice                        %%%%
%%%%%%%%%%%%%%%%%%%%%%%%%%%%%%%%%%%%%%%%%%%%%%%%%%%%%%%%%%%%%%%%%%%%%%%%%%%%
%%%%%%%%%%%%%%%%%%%%%%%%%%%%%%%%%%%%%%%%%%%%%%%%%%%%%%%%%%%%%%%%%%%%%%%%%%%% 
% References and Citations will be updated after second compilation.



\documentclass[a4paper, 10pt, conference]{ieeeconf}      % For A4 Paper
%The Class IEEECONF will be installed by MiKTeX and Texmaker automatically
% All the required Packages will be automatically installed if you have MiKTeX and Texmaker 

\usepackage{amsmath}
\usepackage{amssymb} 
\usepackage{graphics}
\usepackage{epsfig}
\usepackage{mathptmx}
\usepackage[utf8]{inputenc} 
% If You use Texmaker it will automatically download missing packages

\title{\LARGE \bf Plantilla simple para informes de avance proyecto del curso}
% Enter your Title here
% If the title is too long use line breaker (\\) to break into two
% If you do not use line breaker manually, it will be broken automatically by LATEX  

\author{Estudiante Maestría en Ciencias de la Información y las Comunicaciones
\thanks{Si desea hacer comentarios sobre su trabajo use esta línea, otherwise delete it}
\thanks{Give Author Details here. In the following line enter your Mail ID 
                  {\tt\small DDD@humanity.in}}
}

\begin{document}


\maketitle
\thispagestyle{empty}
\pagestyle{empty}



\begin{abstract}

Por favor escriba aquí su resumen.
%%%%%%%%%%%%%%%%%%%%%%%%%%%%%%%%%%%%%%%%%%%%%%%%%%%%%%%%%%%%%

%%%%%%%%%%%%%%%%%%%%%%%%%%%%%%%%%%%%%%%%%%%%%%%%%%%%%%%%%%%%%

\end{abstract}



\section{INTRODUCCIÓN}

Escriba aquí su introducción. Como ejemplo, aquí solo se tiene una sección.

% After the \section command give the section name within brcakets
% If the section name is "Computers", It sholud be given as \section{Computers}

\section{Divida su sección en subsecciones}
% You can add more sections by copying the command \section
\subsection{Subsección Uno}

Aquí un sección se divide en subsecciones. Esta es la subsección 1. 

% After the \subsection command give the subsection name within brackets.
% If the section name is "Digital Computers", 
% It sholud be given as \subsection{Digital Computers}

\subsection{Subsección Dos}

This is sub section 2 of the main section.
You can add any number of subsections in a section.
% Copy the subsection command if you have to add more sub sections

\section{Nueva Sección}

Here you can represent the data as a list of items. 
Use the following format to list your Data.
There are three items given below.

\subsection{Creación de múltiples elementos}

\begin{itemize}

\item
 Este es el primer elemento de la lista. You can delete this line and enter your content here.
 
 
\item

%Delete the Portion Below and enter your text
%%%%%%%%%%%%%%%%%%%%%%%%%%%%%%%%%%%%%%%%%%%%%%%%%%%%%%%%%
Was this the face that launched a thousand ships,
And burnt the topless towers of Ilium?
Sweet Helen, make me immortal with a kiss.
Her lips suck forth my soul, see where it flies.
Come, Helen, come, give me my soul again.
Here will I dwell, for heaven is in these lips,
And all is dross that is not Helena.
I will be Paris, and for love of thee,
Instead of Troy, shall Wittenberg be sacked;
And I will combat with weak Menelaus,
And wear thy colours on my plumed crest,
Yea, I will wound Achilles in the heel,
And then return to Helen for a kiss.
O, thou art fairer than the evening air
Clad in the beauty of a thousand stars,
Brighter art thou than flaming Jupiter,
When he appeared to hapless Semele,
More lovely than the monarch of the sky.
In wanton Arethusa's azur'd arms,
And none but thou shalt be my paramour. 
%%%%%%%%%%%%%%%%%%%%%%%%%%%%%%%%%%%%%%%%%%%%%%%%%%%%%%%%%

% you can type in large amount of data below each item in the list, not just a single line.

\item 

This is the Third item in the list. If you want just two items you can delete this whole section.
% If you want to delete this item, delete the above command \item
% You can have any number of items in the list, just copy the command \item 

\end{itemize}

\section{How to Deal with Equations}

This section gives you a template to deal with equations. 
% Greek alphabets can be used in equations as follows.
% Use \followed by the name of the Greek Alphabet
% Eg: \theta
$$
\epsilon + \mu = \nu \eqno{(1)}
$$
$$
\theta + \delta = \gamma \eqno{(2)}
$$
$$
x^2+y^3=z^4 \eqno{(3)}
$$
If you don't want to number your equations use the 
following syntax.
$$
\epsilon + \mu = \nu
$$
$$
\theta + \delta = \gamma 
$$
$$
x^2+y^3=z^4
$$


\section{One More Section}

%Delete the Portion Below and enter your text

%%%%%%%%%%%%%%%%%%%%%%%%%%%%%%%%%%%%%%%%%%%%
A Book of Verses underneath the Bough,
A Jug of Wine, a Loaf of Bread--and Thou
Beside me singing in the Wilderness--
Oh, Wilderness were Paradise enow! 
%%%%%%%%%%%%%%%%%%%%%%%%%%%%%%%%%%%%%%%%%%%%

\subsection{T
	his is a Sub Section}

%Delete the Portion Below and enter your text

%%%%%%%%%%%%%%%%%%%%%%%%%%%%%%%%%%%%%%%%%%%
I sent my Soul through the Invisible,
Some letter of that After-life to spell:
And by and by my Soul return'd to me,
And answer'd "I Myself am Heav'n and Hell:" 
%%%%%%%%%%%%%%%%%%%%%%%%%%%%%%%%%%%%%%%%%%%%%

\subsection{One More Sub Section}

%Delete the Portion Below and enter your text 

%%%%%%%%%%%%%%%%%%%%%%%%%%%%%%%%%%%%%%%%%%%%%%
Ah, my Belov'ed fill the Cup that clears
To-day Past Regrets and Future Fears:
To-morrow!--Why, To-morrow I may be
Myself with Yesterday's Sev'n Thousand Years. 
%%%%%%%%%%%%%%%%%%%%%%%%%%%%%%%%%%%%%%%%%%%%%%%
\section{Tables and Images}

Now let us learn how to create tables and show images in a Conference paper.   

\subsection{Creating Tables}
Copy the following code segment, if you want to have multiple tables in your article.Be sure to
rename the label every time you copy the code.

%%%%%%%%%%%%%%%%%%%%%%%%%%%%%%%%%%%%%%%%%%%%%%%%%%%%%%%%%%%%%%

\begin{table}[h]
\caption{Enter the Table Name Here}
\label{table_example1}
% If you are copying this code to create multiple tables, Please Change the label name
% Eg: change \label{table_example1} to \label{table_example2} 
\begin{center}
\begin{tabular}{|c||c|}
\hline
Sachin & Dhoni\\
\hline
Kohli & Raina\\
\hline
% Please Enter your data in the above code.
\end{tabular}
\end{center}
\end{table}

%%%%%%%%%%%%%%%%%%%%%%%%%%%%%%%%%%%%%%%%%%%%%%%%%%%%%%%%%%%%%%%%%%%%%%%%%%%%%%%%%%%%%%%%%%



\subsection{Showing Images}
Copy the following code segment, if you want to have multiple images\cite{c1} in your article. Be sure to
rename the label every time you copy the code.
The image will be shown below this text once you uncomment the first line of code given 
below and give the path name of the image to be displayed. If the image is given in the
same folder as the Tex file, just enter the name of the image in the code given below. 
% Here \cite{c1} is used to refer to the reference item with label c1
%%%%%%%%%%%%%%%%%%%%%%%%%%%%%%%%%%%%%%%%%%%%%%%%%%%%%%%%%%%%%%%%%%%%%%%%%%%%%%%%%%%%%%%%%%

   \begin{figure}[thpb]
      \centering
      %\includegraphics[scale=0.7]{fig1.jpg}
      % Please Uncomment the above line of code      
      % Enter the name of your image instead of fig1.jpg 
      % Save the image in the same folder where you have the Tex file
      \caption{Enter Your Caption Here}
      \label{img_label1}
      % If you are copying this code to create multiple images, Please Change the label name
      % Eg: change \label{img_label1} to \label{img_label2}
   \end{figure}
   
%%%%%%%%%%%%%%%%%%%%%%%%%%%%%%%%%%%%%%%%%%%%%%%%%%%%%%%%%%%%%%%%%%%%%%%%%%%%%%%%%%%%%%%%


\section{CONCLUSIONES}

Enter your Conclusions here in this section. 

\addtolength{\textheight}{-12cm}   

%%%%%%%%%%%%%%%%%%%%%%%%%%%%%%%%%%%%%%%%%%%%%%%%%%%%%%%%%%%%%%%%%%%%%%%%%%%%%%%%

\section*{APÉNDICE}

The usual convention is to enter Appendixes before the Acknowledgment.

% The * at the end \section* is to avoid inserting a number to this section

\section*{RECONOCIMIENTOS}

% The * at the end of \section* is to avoid inserting a number to this section

% Enter your acknowledgement here. 
% Delete The Portion below and enter your acknowledgement.

%%%%%%%%%%%%%%%%%%%%%%%%%%%%%%%%%%%%%%%%%%%%%%%%%%%%%%%%%%%%%%%%%%%%%%%
I thank the Almighty God.How numerous you have made your wondrous deeds,
O Lord, our God! And in our plans for us there is none to equal you,
Should I wish to equal or tell them,
They would be too many to recount.
%%%%%%%%%%%%%%%%%%%%%%%%%%%%%%%%%%%%%%%%%%%%%%%%%%%%%%%%%%%%%%%%%%%%%%%



\begin{thebibliography}{99}
% you can enter upto 99 references
% Copy the command \bibitem{cn} to create new references
% Here n is the number of the reference being created

\bibitem{c1} Mahathma Gandhi, The Story of My Experiments with Truth, Navajivan Mudranalaya,   
Ahemadabad, India, 1927.

\bibitem{c2} Mahathma Gandhi, Hind Swaraj or Indian Home Rule, Navajivan Mudranalaya,   
Ahemadabad, India, 1938.

\bibitem{c3}  Mahathma Gandhi, Ethical Religion, Navajivan Mudranalaya,   
Ahemadabad, India, 1968.
 

\end{thebibliography}


\end{document}

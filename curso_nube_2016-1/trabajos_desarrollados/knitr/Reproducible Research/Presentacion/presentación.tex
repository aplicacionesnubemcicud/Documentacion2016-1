%%%%%%%%%%%%%%%%%%%%%%%%%%%%%%%%%%%%%%%%%%%%%%%%%%%%%%%%%%%%%%%%%%%%%%%%%%
 %																		%
 %	Plantilla Latex para presentación del proyecto de curso				%
 %	Programación de Aplicaciones para Internet y la Nube					%
 %																		%
 %	Creada por: Duván Pardo, Wilson López
 % Modificada por: Pedro J. Vargas Barrios								%
 %																		%
 %	Versión: 0.2															%
 %	pedrojvar@gmail.com							%
 %																		% 
 %	Se requieren los archivos  presentación.bbl							% 
 %	El directorio Imagenes que contiene: escudoud.pdf,ECHO_OFF	 		%  
 %																		%
%%%%%%%%%%%%%%%%%%%%%%%%%%%%%%%%%%%%%%%%%%%%%%%%%%%%%%%%%%%%%%%%%%%%%%%%%%

\documentclass[11pt]{beamer}					% Describe el tipo de documento, y el tamaño de la letra del texto
\usepackage[utf8]{inputenc}					% Define codificación para que permita caracteres latinos (acentos)
\usepackage[spanish,activeacute]{babel} 		% Paquete para poder escribir con tildes y otros caracteres especiales

\usepackage{amsmath}							% paquete para expresiones matemáticas
\usepackage{amsfonts}						% paquete para escritura de ecuaciones 
\usepackage{amssymb}							% paquete para caracteres especiales para ecuaciones 

\usepackage{svg}								% Se utiliza para incluir imágenes vectorizadas en el documento (.pdf)
\usepackage{hyperref}						% Para hipervinculos

\usepackage{lmodern}							% http://ctan.org/pkg/lm
\usepackage{listings}						% Para el código fuente
\usepackage{xcolor}							% Para el color en código fuente
\usepackage{graphicx}						% Para incluir imágenes
\graphicspath{{Imagenes/}}					% Directorio de imágenes

\bibliographystyle{apalike} 					% Bibliografia tipo APA

\definecolor{limegreen}{RGB}{50,100,50}		% Definición de color
\lstdefinestyle{base}{						% Para el color en código fuente
	language=C,
	emptylines=1,
	breaklines=true,
	showspaces=fasle,
	showstringspaces=false,
	extendedchars=true,
	basicstyle=\ttfamily\color{black},
	moredelim=**[is][\color{limegreen}]{'}{'},
	moredelim=**[is][\color{blue}]{&}{&},
}				
\lstset{numbers=left, numberstyle=\tiny, stepnumber=1, numbersep=5pt}	% Muestra numeración al lado del código

\mode<presentation>{	
	\usetheme{Frankfurt}		
% Temas: AnnArbor,Antibes, Bergen, Berkeley, Berlin, Boadilla, boxes, CambridgeUS, Copenhagen, Darmstadr, default, Dresden, Frankfurt*, Goettingen, Hannover, LLmenau, JuanLesPins, Luebeck, Madrid, Malmoe, Marburg, Montpellier, PaloAlto, Pittsburgh, Rochester, Singapore, Szeged, Warsaw.	
	\usecolortheme{orchid}	
% Colores:albatross, beaver, beetle, crane, default, dolphin, dove, fly, lily, orchid*, rose, seagull, seahorse, sidebartab, structure, whale, wolverine.	 
}

\logo{\includegraphics[scale=0.04]{escudoud}}
\title{Dinamic Documents \\ --- \\ Reproducible Research}
\author{Estudiante(s): Pedro J. Vargas Barrios}
\institute[UD]{Universidad Distrital Francisco José de Caldas}
\date{\today}

\begin{document}	
	
	\begin{frame}[fragile]							% Diapositiva Presentación
		\titlepage 
		\begin{small}
			''Investigación Reproducible''
		\end{small}
	\end{frame}	

    	\begin{frame}[fragile]							% Diapositiva Tabla de Contenido
		\frametitle{Índice}	
		\tableofcontents
	\end{frame}	

\section{Contexto}
		 \begin{frame}[fragile]						% Primera Diapositiva
			\frametitle{Contexto}
			\begin{huge}
			\begin{center}
				\emph{\textit{Reproducibilidad - Programación literaria}}
			\end{center}
			\end{huge}
		\end{frame}	
	\subsection{Elementos Investigación Reproducible}	
		\begin{frame}[fragile]						% Segunda Diapositiva
				\frametitle{Investigación Reproducible}
				\begin{block}{Jon Claerbout - Universidad de Stanford - 2009}
''El producto final de la
investigación no es sólo el papel en sí , sino que también es
el entorno computacional completo utilizado para producir los resultados en el documento"
				\end{block}
				\begin{block}{Elementos para garantizar la reproducibilidad}	
				
				\begin{itemize}
				\item El artículo o informe técnico.
				\item El programa de cómputo o paquetería de trabajo.
				\item El experimento numérico o flujo de trabajo, incluyendo los códigos y la secuencia de instrucciones para generar los resultados.
				\item Los datos empleados.
				\item Los resultados del experimento, como figuras y datos.
				\end{itemize}
				

				\end{block}			
			\end{frame}	
			
\section{Hacia la Investigación Reproducible}	
		 \begin{frame}[fragile]
			\frametitle{Etapas destacadas}
			\begin{huge}
			\begin{center}
				\emph{\textit{Hacia la Investigación Reproducible}}
			\end{center}
			\end{huge}
		\end{frame}		
		   		
    		\subsection{Etapas destacadas}			
			\begin{frame}[fragile]
		
			
			\begin{block}{Etapas destacadas}
			
			\begin{itemize}
			    \item (1983) - primera implementación de herramienta llamada WEB para programación literaria, poco adecuada para análisis de datos.
				\item (1983) - Knuth plantea el paradigma de programación literaria, muy relacionado con la investigación reproducible.
				\item (1994) -  implementación de herramienta Noweb para programación literaria, poco adecuada para análisis de datos.
				\item (2002) - Sweave fue una de las primeras implementaciones para tratar los documentos dinámicos en R.
				\item (2009) - Roger Peng publica en revista de Bioestadística criterios de reproducibilidad de su investigación.
				\item (2015) - Herramienta roxygen2, implementación de R Doxygen
				\item (2012 -2015) - El paquete knitr se basa en las ideas de las herramientas anteriores con un framework rediseñado.

				\end{itemize}

				\end{block}

			\end{frame}
			
			\section{Buenas y malas prácticas en Investigación Reproducible}	
		 \begin{frame}[fragile]
		 
			\frametitle{Recomendaciones}
			\begin{huge}
			\begin{center}
				\emph{\textit{Buenas y malas prácticas en Investigación Reproducible}}
			\end{center}
			\end{huge}
		\end{frame}		
		\subsection{Lista de buenas y malas prácticas}			
		 \begin{frame}[fragile]						% Primera Diapositiva
			\frametitle{Lista de buenas y malas prácticas}
En el capítulo se menciona que la clave a tener en cuenta
para la investigación reproducible es que otras personas puedan
ser capaces de reproducir nuestros resultados, por lo tanto, se
debe hacer todo lo posible para hacer que la computación sea portable.			
			
	\begin{itemize}
			    \item Gestionar todos los archivos de origen en el mismo directorio y utilizar rutas relativas siempre que sea posible.
				\item No cambiar el directorio de trabajo después de que la ejecución ha comenzado.
				\item Compilar los documentos en una sesión de R limpia.
				\item Evitar los comandos que requieren la interacción humana.
				\item Evitar las variables de entorno para el análisis de datos.
				\item Adjuntar sessioninfo() y las instrucciones de cómo compilar el documento.
				
				\end{itemize}
		\end{frame}
		
	\section{Barreras}	
		 \begin{frame}[fragile]
			\frametitle{Barreras}
			\begin{huge}
			\begin{center}
				\emph{\textit{Barreras}}
			\end{center}
			\end{huge}
		\end{frame}		
		\subsection{Lista de obstáculos}	
		 \begin{frame}[fragile]	
		 \frametitle{Lista de obstáculos}
		Se mencionan en el documento algunas barreras prácticas que existen a pesar de todas las ventajas de la Reproducible Research, y se menciona a continuación una lista no exhaustiva de ellas.
		
		\begin{itemize}
			    \item Los datos pueden ser enormes
				\item Confidencialidad de los datos.
				\item Versión del software y configuración.
				\item La competencia.

				
		\end{itemize}
		
		\end{frame}
		

							
\section{Bibliografía}	
%BIBLIOGRAFIA 
	\begin{frame}[fragile]
		\frametitle{Bibliografía} 		
        \bibliography{Biblio} 	%Para que aparezca toda la bibliografia que citamos en el documento Biblio.bib, Informe.bbl.
       	\nocite{*}				%Para que aparezca toda la bibliografia que NO citamos en el documento, pero que utilizamos.
	\end{frame}		
				
\end{document}

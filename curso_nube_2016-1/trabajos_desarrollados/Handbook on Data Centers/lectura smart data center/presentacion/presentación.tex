%%%%%%%%%%%%%%%%%%%%%%%%%%%%%%%%%%%%%%%%%%%%%%%%%%%%%%%%%%%%%%%%%%%%%%%%%%
 %																		%
 %	Plantilla Latex para presentación del proyecto de curso				%
 %	Programación de Aplicaciones para Internet y la Nube					%
 %																		%
 %	Creada por: Duván Pardo, Wilson López								%
 %																		%
 %	Versión: 0.2															%
 %	Dapardoc@gmail.com ; Wilrilo@gmail.com								%
 %																		% 
 %	Se requieren los archivos  presentación.bbl							% 
 %	El directorio Imagenes que contiene: escudoud.pdf,ECHO_OFF	 		%  
 %																		%
%%%%%%%%%%%%%%%%%%%%%%%%%%%%%%%%%%%%%%%%%%%%%%%%%%%%%%%%%%%%%%%%%%%%%%%%%%

\documentclass[11pt]{beamer}					% Describe el tipo de documento, y el tamaño de la letra del texto
\usepackage[utf8]{inputenc}					% Define codificación para que permita caracteres latinos (acentos)
\usepackage[spanish,activeacute]{babel} 		% Paquete para poder escribir con tildes y otros caracteres especiales

\usepackage{amsmath}							% paquete para expresiones matemáticas
\usepackage{amsfonts}						% paquete para escritura de ecuaciones 
\usepackage{amssymb}							% paquete para caracteres especiales para ecuaciones 

\usepackage{svg}								% Se utiliza para incluir imágenes vectorizadas en el documento (.pdf)
\usepackage{hyperref}						% Para hipervinculos

\usepackage{lmodern}							% http://ctan.org/pkg/lm
\usepackage{listings}						% Para el código fuente
\usepackage{xcolor}							% Para el color en código fuente
\usepackage{graphicx}						% Para incluir imágenes
\graphicspath{{Imagenes/}}					% Directorio de imágenes

\bibliographystyle{apalike} 					% Bibliografia tipo APA

\definecolor{limegreen}{RGB}{50,100,50}		% Definición de color
\lstdefinestyle{base}{						% Para el color en código fuente
	language=C,
	emptylines=1,
	breaklines=true,
	showspaces=fasle,
	showstringspaces=false,
	extendedchars=true,
	basicstyle=\ttfamily\color{black},
	moredelim=**[is][\color{limegreen}]{'}{'},
	moredelim=**[is][\color{blue}]{&}{&},
}				
\lstset{numbers=left, numberstyle=\tiny, stepnumber=1, numbersep=5pt}	% Muestra numeración al lado del código

\mode<presentation>{	
	\usetheme{Frankfurt}		
% Temas: AnnArbor,Antibes, Bergen, Berkeley, Berlin, Boadilla, boxes, CambridgeUS, Copenhagen, Darmstadr, default, Dresden, Frankfurt*, Goettingen, Hannover, LLmenau, JuanLesPins, Luebeck, Madrid, Malmoe, Marburg, Montpellier, PaloAlto, Pittsburgh, Rochester, Singapore, Szeged, Warsaw.	
	\usecolortheme{orchid}	
% Colores:albatross, beaver, beetle, crane, default, dolphin, dove, fly, lily, orchid*, rose, seagull, seahorse, sidebartab, structure, whale, wolverine.	 
}

\logo{\includegraphics[scale=0.04]{escudoud}}
\title{Informe de lectura de investigación Smart Data Center \\ --- \\ Smart Data center}
\author{Estudiante: Wilson López}
\institute[UD]{Universidad Distrital Francisco José de Caldas}
\date{\today}

\begin{document}	
	
\begin{frame}[fragile]							% Diapositiva Presentación
		\titlepage 
	
\end{frame}	

	\begin{frame}[fragile]							% Diapositiva Tabla de Contenido
		\frametitle{Índice}	
		\tableofcontents
\end{frame}	



\section{Resumen}
\begin{frame}[fragile]						% Segunda 
	\frametitle{Resumen}	
	\begin{block}{}
		\begin{itemize}
			\item Optimización de consumo energetico en Data Center.
			
			
			\item Compra y almacenamiento eficiente de energía.
		\end{itemize}
	\end{block}

	  	
				

		

\end{frame}		

\section{Contribuci'on de la investigaci'on}
\begin{frame}[fragile]						% Segunda 
	\frametitle{Contribuci'on de la investigaci'on}					
	\begin{center}
	\begin{itemize}
		\item El artículo SI contribuye:  método para determinar la compra y uso eficiente de la energía por medio de UPS para data center\\
		
		\item El algoritmo  “smart data center“ utiliza algoritmos de optimización Lyapunov. 
		
		\item Implementación del algoritmo en CECAD para mejorar consumo energético y vida útil de las UPS consumo CECAD aproximado 25KVA. 	
		
	\end{itemize}
	\end{center}
\end{frame}	




\section{Evidencias de soporte}
\begin{frame}[fragile]						% Segunda 
	\frametitle{Evidencias de soporte}					
	\begin{center}
	\begin{itemize}
		\item \href{http://www.sciencedirect.com/science/article/pii/S0378779604001257}{\textit{Electricity market price spike forecast with data mining techniques}}. escrito por X. Lu, Xin, Z. Y. Dong, y X. Li. realizaron un estudio de   los precios de la demanda de energía para predecir las futuras demandas de energía 
		
		\item En los articulos 
			\href{http://sameekhan.seecs.nust.edu.pk/pub/B_K_2013_FGCS.pdf}{A Taxonomy and Survey on Green Data Center Networks},	
			\href{http://ieeexplore.ieee.org/xpl/login.jsp?tp=&arnumber=5683561&url=http\%3A\%2F\%2Fieeexplore.ieee.org\%2Fiel5\%2F5682081\%2F5683069\%2F05683561.pdf\%3Farnumber\%3D5683561}{GreenCloud: A Packet-level Simulator of Energyaware Cloud Computing Data Centers} y 	
			 \href{http://www.hpl.hp.com/techreports/2007/HPL-2007-194.pdf}{No ‘power’ struggles: Coordinated multi-level power management for the data center} ; Se estudian la administración de energía de los centros de datos
		
	\end{itemize}
					
	\end{center}
\end{frame}		
				


\begin{frame}[fragile]						% Segunda 
	\frametitle{Evidencias de soporte}					
	\begin{center}
		Los siguientes artículos estudian el almacenamiento de energía en data center:
		
		\begin{itemize}
			
				
		\item \textit{“	\href{http://www.cse.psu.edu/~buu1/papers/ps/asplos12.pdf}{Leveraging stored energyfor handling power emergencies in aggressively provisioned datacenters}”} de. S. Govindan, D. Wang, A. Sivasubramaniam, y B. Urgaonkar
			
		
			
	\item \textit{“\href{http://www.fang.ece.ufl.edu/mypaper/globecom11guoy.pdf}{Cutting down electricity cost in internet data centers by using energy storage}”}  de Y. Guo, Z. Ding, Y. Fang, and D. Wu
			
			
	\item \textit{“\href{http://www.cse.psu.edu/~buu1/papers/ps/sigmetrics11.pdf}{Optimal power cost management using stored energy in data centers}”} de R. Urgaonkar, B. Urgaonkar, M. l. J. Neely, y A. Sivasubramaniam

\item \textit{“	\href{http://www.cse.psu.edu/~buu1/papers/ps/sigmetrics12.pdf}{Energy storage in datacenters: what, where, and how much?}”} de  . D. Wang, C. Ren, A. Sivasubramaniam, B. Urgaonkar, and H. Fathy
		\end{itemize}
		
	\end{center}
\end{frame}


\begin{frame}[fragile]						% Segunda 
	\frametitle{Evidencias de soporte}					
	\begin{center}
		Contratación a largo plazo de energía hablan:
		 \begin{itemize}	
		 	
		 	
		 	\item  M. He, S. Murugesan en “\href{http://arxiv.org/pdf/1008.3932.pdf}{Multiple timescale dispatch and scheduling for stochastic reliability in smart grids with wind generation integration}
		 	
		 	
		 	
		 	\item	L. Huang en \href{https://arxiv.org/pdf/1112.0623.pdf}{Optimal power procurement and demand response with quality-of-usage guarantees}
		 	
		 	\item J. Nair en  \href{http://users.cms.caltech.edu/~adamw/papers/Wind-preprint.pdf}{Energy procurement strategies in the presence of intermittent sources}
		 	
		 	
		 	Multi-period optimal procurement and demand responses in the presence of uncrtain supply
		 	\item L. Jiang en  “\href{http://smart.caltech.edu/papers/multiperiod.pdf}{Multi-period optimal procurement and demand responses in the presence of uncrtain supply}"
		 	
		 	
		 \end{itemize}	
		
	\end{center}
\end{frame}



\section{Comentarios de 'arbitro:}
\begin{frame}[fragile]						% Segunda 
	\frametitle{Comentarios de 'arbitro:}					
	\begin{center}
		
		\begin{itemize}
			 
		\item	El artículo realiza un buen estudio del estado del arte del tema.
			
		\item	Buen soporte matemático sobre el algoritmo que desarrollan.
			
		\item	Falta Implementación del algoritmo para comprobar su funcionamiento 
		\end{itemize}
	\end{center}
\end{frame}


			
\section{Bibliografía}	
%BIBLIOGRAFIA 
\begin{frame}[fragile]
		\frametitle{Bibliografía} 		
        \bibliography{presentación} 	%Para que aparezca toda la bibliografia que citamos en el documento Biblio.bib, Informe.bbl.
       	\nocite{*}				%Para que aparezca toda la bibliografia que NO citamos en el documento, pero que utilizamos.
\end{frame}		
			
			\end{document}

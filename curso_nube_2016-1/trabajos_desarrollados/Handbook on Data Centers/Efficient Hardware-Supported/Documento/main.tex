%Tipo de Documento [Conferencia]

\documentclass[conference]{IEEEtran}

%BIBLIOTECAS

% Este paquete se utiliza para generar texto o graficas de relleno.
%\usepackage{blindtext, graphicx}
%Biblioteca para graficas
\usepackage{graphicx}
%Biblioteca para lectura de caracteres ortográficos (tildes..etc. ) 
\usepackage[utf8]{inputenc}
%Biblioteca para graficos vectrizados.svg 
\usepackage{svg}
%Biblioteca para enumerar figuras tablas.. etc en español 
\usepackage[spanish, es-tabla]{babel}


%INICIO DEL DOCUMENTO
\begin{document}


% TITULO DEL PAPER
\title{Control de lectura - Hardware eficiente apoyado en mecanismos de sincronización
para muchos núcleos
}


% NOMBRE DE LOS AUTORES
\author{\IEEEauthorblockN{Pedro J. Vargas B.}

%DATOS DE CADA AUTOR EN COLUMNAS

%AUTOR 1	
\IEEEauthorblockA{Maestría en Ciencias de la Información y las Comunicaciones\\ % \\ se utiliza para pasar al siguiente renglón
Ingeniero de sistemas\\
Universidad Distrital\\
Bogotá\\
Email: pjvargasb@correo.udistrital.edu.co}
}


%TITULO
\maketitle

%Iniciar Abstract
\begin{abstract}
	El impacto de las necesidades de cómputo ha llevado a que se avance en el desarrollo de la nueva generación de procesadores, los 
	manycore, que se caracterízan porque tienen una cantidad de núcleos amplia, además de que entregan un alto grado de paralelismo, debido a  la gran cantidad de núcleos que poseen.En el cápitulo se expresan las característcas de sincronización, de trabajo y las ventajas que proporciona este tipo de procesadores para los nuevos enfoques tecnológicos.
\end{abstract}


%Iniciar Palabras Clave Formato IEEE
\begin{IEEEkeywords}
Hardware, Sincronización, núcleos.
\end{IEEEkeywords}


%SECCIÓN 1. INTRODUCCIÓN 
\section{Introducción}
Los centros de datos están evolucionando al acoger aplicaciones distribuidas y en paralelo
tales como la computación en nube, la transmisión de vídeo o las redes sociales. Estas nuevas aplicaciones
exigen no sólo más capacidad de almacenamiento o ancho de banda de red, sino también mayor
rendimiento, por lo que requiere procesadores manycore como bloques de construcción
de cada nodo o servidor computacional.\\
Como se están integrando cada vez más núcleos en el chip , arquitecturas manycore han surgido como la próxima generación de multicores .Los Manycores son sistemas especialmente adaptados a la explotación de rendimiento masivo mediante la incorporación de muchos
unidades de computación simples y de baja frecuencia. Este cambio de paradigma conduce a las cargas de trabajo en paralelo con cada vez un mayor número de hilos que necesitan comunicarse y sincronizarse entre ellos , por lo general dependen de un solo
dominio de memoria compartida por servidor.\\
En este contexto las implementaciones convencionales de las operaciones de sincronización , tales
como barrier (barrera) y locks (bloqueos) , hacen uso de las variables compartidas que se actualizan de forma atómica .
En particular , al considerar barreras globales y locks contendió altos ( es decir, una
cantidad significativa de hilos que solicitan el lock al mismo tiempo ), sin el apoyo adecuado de hardware, las implementaciones típicas basadas en software no pueden proporcionar buena escalabilidad como el número de núcleos aumenta.

%PARA INSERTAR FIGURAS, IMAGENES 
	%	\begin{figure}[h]
	%		\centering
	%		\includegraphics[width=0.2\textwidth]{escudoudblancoynegro}
	%		\caption{Escudo Universidad Distrital Francisco Jóse de Caldas}
	%	\end{figure}
%SECCIÓN 2.  
\section{Las Lineas de Tecnologias G}
	las lineas G ya se han integrado con éxito en un sustrato de silicio con el fin de
permitir las comunicaciones de velocidad de la luz punto a punto. Chang et al.  y José et
Alabamam mostraron principios de circuitos de punto a punto que permiten la transmisión de línea , velocidad de onda similar
por 10 mm de interconexión. No obstante , esta implementación inicial sufre
los gastos generales significativos en términos de disipación de energía. Un gran esfuerzo
se ha dedicado a superar esta y otras limitaciones . Por ejemplo, se ha extendido las 
G -Lines para apoyar la difusión, multi-drop y transmisiones bidireccionales. esta contribución
permite tanto la baja latencia y capacidad multi-drop en una línea de transmisión con
disipación de baja potencia. Sin embargo, sus resultados siguen mostrando varias cuestiones integración.


%SUBSECCIÓN A.  TITULO SUBSECCIÓN
%\subsection{Hardware de sincronizacion Barrier}
\section{Hardware de sincronizacion Barrier}
Un Barrier o Barrera es una primitiva de sincronización que permite que varios procesos o hilos puedan
esperar en un punto de ejecución especial, hasta que todos ellos han llegado a este antes de que cualquiera
de ellos pueda continuar. Un ejemplo típico de su uso es la utilización de barreras para separar
las diferentes fases que se encuentran comúnmente en aplicaciones paralelas. Al hacerlo, el
programador se asegura de que la segunda fase no se inicia hasta que todos los procesos o hilos
de la aplicación han completado la primera.
En el contexto de los sistemas que implementan un modelo de programación de memoria compartida, con el advenimiento de las arquitecturas manycore, nuevos cambios están surgiendo para proporcionar una implementación Barrier eficiente. Esto se debe principalmente al hecho de que de manera diferente al multiprocesador, aplicaciones clásicas que tienen como objetivo el paralelismo coarse-grained, las aplicaciones Manycore tienden a explotar el paralelismo de fine-grained, y por lo tanto , pueden ser muy sensibles a rendimiento del Barrier.


\section{El mecanismo de sincronizacion GBarrier}
Se presenta una propuesta para construir una infraestructura de hardware eficiente
para la sincronización de Barrier en el contexto de los servidores manycore. Para ello, se comienza por
la descripción de la arquitectura de la red en el chip dedicado. Por simplicidad, se dá la explicación asumiendo la tecnología G -Lines con
la técnica S-CSMA. Como caso de estudio, se elige un servidor con una interconexión de datos 2D de malla
red con hileras R de C núcleos cada uno ( para un total de N = R * núcleos C ). 



%SUBSECCIÓN B.  TITULO SUBSECCIÓN
\subsection{Architectura de red dedicada On-Chip}
El mecanismo GBarrier se basa en una red en el chip dedicado. Para simplificar, se concentran en una versión de la propuesta de
red que proporciona apoyo a una de las barreras. la infraestructura GBarrier se compone de dos tipos de componentes. G- Lines ( horizontal y vertical), que se utilizan para transmitir las señales requeridas por el protocolo de sincronización; y controladores ( M y S ) , que en realidad implementan el protocolo de sincronización.\\
Cada G -Line es un cable que permite la transmisión de un bit de información a través de una de las dimensiones del chip en un solo ciclo de reloj. la propuesta de red basada en G -Line emplea dos G- Líneas por barrera (Barrier) para cada fila y dos más
para la primera columna. De esta manera, para cualquier diseño 2D de malla con hileras R y las columnas C ,
el número total de G- Líneas por la barrera que serían necesarios es igual a 2 * ( R 1 )
(por ejemplo, 10 G -Lines para el servidor de 16 núcleos).

\subsection{protocolo de sincronizacion}
El protocolo de sincronización implementado en la parte superior de la red basada en G -Line previamente
descrito se basa en el intercambio de mensajes de 1 bit (señales) entre los
controladores maestro y esclavo , y el uso de la técnica de S- CSMA en los controladores maestro
 para contar el número de señales transmitidas a través de cada G - Line. En la
propuesta, cada sincronización barrera se lleva a cabo mediante el uso de un protocolo de dos fases :
la fase de cuenta y la fase de liberación. La primera fase comienza cuando el primer hilo
llega a la barrera y termina cuando el último alcanza la barrera. A continuación, la segunda
fase , en la que todas las discusiones que participan en el Barrier, se les ordena reanudar
ejecución , se inicia una  interacción exacta entre los hilos , G -Lines y controladores.

\subsection{Problemas de programabilidad}
	 
%SECCIÓN 2. CONCLUSIÓN 	

El mecanismo GBarrier 
propuesto esta destinado a ser
 utilizado por los programadores en un medio transparente. 
 Por esta razon, se propone proporcionar un metodo de barrera de 
 nivel de biblioteca especial GL Barrier que encapsula 
 la funcionalidad de GBarrier y que podrian utilizarse
  en aplicaciones paralelas para hacer frente a operaciones de barrera.
   Este metodo de barrera utiliza un registro de
    1 bit especial, 
    llamado bar reg,
    que notifica la llegada a la barrera estableciendo su valor a uno. El registro bar reg necesita tantos bits como el número de GBarriers
proporcionado por el hardware (un bit por barrera). De esta manera, varias operaciones de barrera
de diferentes conjuntos de núcleos (los hilos en cada juego se ejecuta una aplicación)
podrían tener lugar simultáneamente. De este modo, el archivo de registro de cada núcleo debe estar
aumentada con el registro bar reg y la interacción entre los controladores y éstos
registros deben estar habilitados, el cambio en los controladores siempre que los registros bar reg
están escritos, y la reposición de los registros y la desconexión de los controladores una vez
todos los controladores han terminado la sincronización.
	
	\section{Implementacion de tecnologias}
	
Había varias razones que llevaron a utilizar la tecnología G -Lines para desarrollar
mecanismos de sincronización de barreras en los servidores manycore . En primer lugar, la conectividad
patrón utilizado para desplegar la red de GBarrier dedicado es
basado enlaces unidimensionales de 1 bit que se integran perfectamente en el concepto de
G -Lines . En segundo lugar, los resultados prometedores que podrían lograrse utilizando esta tecnología
en términos de sobrecarga área marginal y la disipación de energía mínimo . Esto llevaria a que se
se obtendrían con ello aún más bajas implicaciones para la zona On-Chip. esta area marginal
de cabecera tendrá también un impacto insignificante sobre la disipación de energía. 

	
	


	\section{Conclusiónes}
%SECCIÓN DE APENDICES  

Se analizo GBarrier para entender mejor su impacto potencial sobre la
actuación en procesadores manycore. En primer lugar , se discutieron algunas consideraciones tomadas al usar
ambos tecnologicas lines-G estándar para implementar una propuesta que aproveche mejor el poder de computo. Se presentaron
implementaciones, mostrar sus posibles contribuciones a las prestaciones en términos de
algunas estadísticas primas importantes como el área dedicada disipación de potencia en el chip ,
máxima velocidad de funcionamiento y latencias mínimas para completar una barrera operación.


\end{document}